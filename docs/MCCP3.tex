%package list
\documentclass{article}
\usepackage[top=3cm, bottom=3cm, outer=3cm, inner=3cm]{geometry}
\usepackage{graphicx}
\usepackage{url}

%% \usepackage{cite}
\usepackage{hyperref}
\usepackage{array}
\usepackage{multicol}
\newcolumntype{x}[1]{>{\centering\arraybackslash\hspace{0pt}}p{#1}}
\usepackage{natbib}
\usepackage{pdfpages}
\usepackage{multirow}
\usepackage{float}
\usepackage[normalem]{ulem}
\useunder{\uline}{\ul}{}


%%%%%%%%%%%%%%%%%%%%%%%%%%%%%%%%%%%%%%%%%%%%%%%%%%%%%%%%%%%%%%%%%%%%%%%%%%%%
%%%%%%%%%%%%%%%%%%%%%%%%%%%%%%%%%%%%%%%%%%%%%%%%%%%%%%%%%%%%%%%%%%%%%%%%%%%%
\newcommand{\csemail}{vmachacaa@unsa.edu.pe}
\newcommand{\csdocente}{Vicente Machaca Arceda}
\newcommand{\cscurso}{Algoritmos y Estructura de Datos}
\newcommand{\csuniversidad}{Universidad Nacional de San Agustín}
\newcommand{\csescuela}{Maestría en Ciencia de la Computación}
\newcommand{\cspracnr}{03}
\newcommand{\cstema}{--}
%%%%%%%%%%%%%%%%%%%%%%%%%%%%%%%%%%%%%%%%%%%%%%%%%%%%%%%%%%%%%%%%%%%%%%%%%%%%
%%%%%%%%%%%%%%%%%%%%%%%%%%%%%%%%%%%%%%%%%%%%%%%%%%%%%%%%%%%%%%%%%%%%%%%%%%%%


\usepackage[english,spanish]{babel}
\usepackage[utf8]{inputenc}
\AtBeginDocument{\selectlanguage{spanish}}
\renewcommand{\figurename}{Figura}
\renewcommand{\refname}{Referencias}
\renewcommand{\tablename}{Tabla} %esto no funciona cuando se usa babel
\AtBeginDocument{%
	\renewcommand\tablename{Tabla}
}

\usepackage{fancyhdr}
\pagestyle{fancy}
\fancyhf{}
\setlength{\headheight}{30pt}
\renewcommand{\headrulewidth}{1pt}
\renewcommand{\footrulewidth}{1pt}
\fancyhead[L]{\raisebox{-0.2\height}{\includegraphics[width=3cm]{img/logo_unsa.jpg}}}
\fancyhead[C]{}
\fancyhead[R]{\fontsize{7}{7}\selectfont	\csuniversidad \\ \csescuela \\ \textbf{\cscurso} }
\fancyfoot[L]{MSc. Vicente Machaca}
\fancyfoot[C]{\cscurso}
\fancyfoot[R]{Página \thepage}

\begin{document}
	
	\vspace*{10px}
	
	\begin{center}	
		\fontsize{17}{17} \textbf{ Práctica \cspracnr}
	\end{center}
	%\centerline{\textbf{\underline{\Large Título: Informe de revisión del estado del arte}}}
	%\vspace*{0.5cm}
	

	\begin{table}[h]
		\begin{tabular}{|x{4.7cm}|x{4.8cm}|x{4.8cm}|}
			\hline 
			\textbf{DOCENTE} & \textbf{CARRERA}  & \textbf{CURSO}   \\
			\hline 
			\csdocente & \csescuela & \cscurso    \\
			\hline 
		\end{tabular}
	\end{table}	
	
	
	\begin{table}[h]
		\begin{tabular}{|x{4.7cm}|x{4.8cm}|x{4.8cm}|}
			\hline 
			\textbf{PRÁCTICA} & \textbf{TEMA}  & \textbf{DURACIÓN}   \\
			\hline 
			\cspracnr & Quadtree y Octree & 3 horas   \\
			\hline 
		\end{tabular}
	\end{table}
	
	
	\section{Datos de los estudiantes}
	\begin{itemize}
		\item Grupo: V
		\item Integrantes: 
		\begin{itemize}
			\item Angel Yvan Choquehuanca Peraltilla
			\item Estefany Pilar Huaman Colque
            \item Eduardo Diaz Huayhuas
            \item Gustavo Raul Manrique Fernandez
		\end{itemize}		
	\end{itemize}
	
	
 
	
	%\clearpage
	%\bibliographystyle{apalike}
	%\bibliographystyle{IEEEtranN}
	%\bibliography{bibliography}
		

    \section{Introducción}
    En el caso de árboles binarios, si los árboles están sesgados, se vuelven computacionalmente ineficientes para realizar operaciones en ellos.
    Esta es la motivación detrás de asegurarse de que los árboles no estén sesgados. De ahí la necesidad de árboles binarios balanceados.

    
\section{Marco Teorico}
\subsection{Quadtree}

Un quadtree es una estructura de datos de árbol en la que cada nodo interno tiene exactamente cuatro hijos. Los Quadtrees son el análogo bidimensional de los octrees y se usan con mayor frecuencia para dividir un espacio bidimensional subdividiéndolo recursivamente en cuatro cuadrantes o regiones. Los datos asociados con una celda de la hoja varían según la aplicación, pero la celda de la hoja representa una "unidad de información espacial interesante".

Las regiones subdivididas pueden ser cuadradas o rectangulares, o pueden tener formas arbitrarias. Esta estructura de datos fue nombrada quadtree por Raphael Finkel y JL Bentley en 1974.Una partición similar también se conoce como Q-tree . Todas las formas de quadtrees comparten algunas características comunes:

\begin{itemize}
    \item Descomponen el espacio en células adaptables.
    \item Cada celda (o balde) tiene una capacidad máxima. Cuando se alcanza la capacidad máxima, el cubo se divide
    \item El directorio del árbol sigue la descomposición espacial del quadtree.
\end{itemize}

\subsection{Tipos}

Los Quadtrees pueden clasificarse según el tipo de datos que representan, incluidas áreas, puntos, líneas y curvas. Los Quadtrees también pueden clasificarse según si la forma del árbol es independiente del orden en que se procesan los datos. Los siguientes son tipos comunes de quadtrees.

\subsubsection{Región Quadtree}

El árbol cuadrangular de la región representa una partición del espacio en dos dimensiones al descomponer la región en cuatro cuadrantes iguales, subcuadrantes, etc., con cada nodo hoja que contiene datos correspondientes a una subregión específica. Cada nodo del árbol tiene exactamente cuatro hijos o no tiene hijos (un nodo hoja). La altura de los cuadrantes que siguen esta estrategia de descomposición (es decir, subdividir subcuadrantes siempre que haya datos interesantes en el subcuadrante para los que se desea un mayor refinamiento) es sensible y depende de la distribución espacial de áreas interesantes en el espacio que se está descomponiendo. La región quadtree es un tipo de trie.

\begin{figure}[H]
\centering
\includegraphics[width=0.9\textwidth]{img/quad1.png}
\caption{Ejemplo de Segmentacion en Regiones Quadtree}
\end{figure}

\subsubsection{Punto Quadtree}

El árbol cuádruple de puntos  es una adaptación de un árbol binario utilizado para representar datos puntuales bidimensionales. Comparte las características de todos los quadtrees, pero es un verdadero árbol, ya que el centro de una subdivisión siempre está en un punto. A menudo es muy eficiente para comparar puntos de datos ordenados bidimensionales, que generalmente operan en tiempo O (log n) . Vale la pena mencionar los quadtrees de puntos por su integridad, pero han sido superados por árboles k -d como herramientas para la búsqueda binaria generalizada.

\begin{figure}[H]
\centering
\includegraphics[width=0.4\textwidth]{img/quadv3.jpg}
\caption{Punto Quadtree}
\end{figure}

\subsubsection{Quadtree de región puntual (PR)}

Los quadtrees de región puntual (PR) son muy similares a los quadtrees de región. La diferencia es el tipo de información almacenada sobre las células. En un quadtree de región, se almacena un valor uniforme que se aplica a toda el área de la celda de una hoja. Las celdas de un quadtree PR, sin embargo, almacenan una lista de puntos que existen dentro de la celda de una hoja. Como se mencionó anteriormente, para los árboles que siguen esta estrategia de descomposición, la altura depende de la distribución espacial de los puntos. Como el quadtree de puntos, el quadtree PR también puede tener una altura lineal cuando se le da un conjunto "malo"

\begin{figure}[H]
\centering
\includegraphics[width=0.7\textwidth]{img/quadPR.png}
\caption{Quadtree de región Puntual}
\end{figure}

\subsubsection{Borde quadtree}
Los quadtrees de borde  (al igual que los quadtrees PM) se utilizan para almacenar líneas en lugar de puntos. Las curvas se aproximan subdividiendo las celdas a una resolución muy fina, específicamente hasta que hay un solo segmento de línea por celda. Cerca de las esquinas / vértices, los cuarteles de bordes continuarán dividiéndose hasta que alcancen su nivel máximo de descomposición. Esto puede resultar en árboles extremadamente desequilibrados que pueden frustrar el propósito de la indexación.

\subsection{Algunos usos comunes de Quadtrees}
\begin{itemize}
    \item Procesamiento de imágenes
    \item Generación de mallas
    \item Indexación espacial , consultas de ubicación de puntos y consultas de rango.
    \item Detección de colisiones eficiente en dos dimensiones
    \item Ver selección de datos de terreno frustum
    \item Almacenar datos escasos, como información de formato para una hoja de cálculo [12] o para algunos cálculos matriciales.
    \item Solución de campos multidimensionales (dinámica de fluidos computacional , electromagnetismo)
    \item Programa de simulación Game of Life de Conway . 
    \item Estimación estatal 
    \item Los quadtrees también se utilizan en el área del análisis de imágenes fractales.
    \item Conjuntos disjuntos máximos
\end{itemize}   

\section{Metodologia y Desarrollo}

        
\section{Resultados}




\section{Conclusiones}


	
	
	\end{document}
